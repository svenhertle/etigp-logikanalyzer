%tex
%% $Id: arbeit.tex 78 2009-07-30 19:49:33Z miracle $
% Bitte verwenden Sie pdflatex.
% Bitte beachten: es mu"s hier eine Sprachangabe gemacht werden.
% Verwenden Sie im Zweifelsfall ngerman.
\documentclass[IN,ngerman,utf8,12pt]{tumbook}
% Optionen f"ur tumbook:

%  Fakult"aten:
%   NEUTRAL -- ohne Logo und Bezeichnung
%   AR   Architektur            BV   Bauingenieur- und Vermessungswesen
%   CH   Chemie                 EI   Elektro- und Informationstechnik
%   IN   Informatik             MA   Mathematik
%   MED  Medizin                MW   Maschinenwesen
%   PH   Physik                 SE   TUM School of Education
%   SP   Sportwissenschaft      WI   Wirtschaftswissenschaften
%   WZW  Wissenschaftszentrum Weihenstephan

%   Sprachen (f"ur Silbentrennung): ngerman, english (siehe babel)
%   Schriftgr"o"sen: 9pt, 10pt, 12pt

\makeindex

% Die klein geschriebenen \title, \author und \date sind obligatorisch.
\Seminar{ETI-Großpraktikum}
\Semester{SS 2013}
\title{Entwicklung eines Logic Analyzers in VHDL}
\Untertitel{ETI-GP 03}
\Themensteller{Georg Acher}
%\Autorenadresse{Anschrift und Telefonnummer des Autors}
%\Matrikelnummer{Matrikelnr. ggf. mit \, gruppiert}
%\Fachsemester{Anzahl der Fachsemester des Autors}
\Abgabetermin{10. August 2013}
\author{Sven Hertle, Markus Engel, Thomas Czok}
%\date{Ort und Datum (f\"ur ehrenw\"ortliche Erkl\"arung)}
\begin{document}
\maketitle%                  Erzeugt die Titelseite
\tableofcontents%            Erzeugt das Inhaltsverzeichnis

\chapter{Einleitung}
Wem danken wir?

\chapter{Begriffsdefinition}
Was ist ein Logic Analyzer? Wie sieht ein professioneller aus? Was kann der alles?

\chapter{Vorgaben}
Welche Hardware haben wir zur Verfügung? Wie sind die Startbedingungen?

\chapter{Spezifikation}
Beschreibung unserer Zielsetzung? Was wollen wir umsetzen?

\chapter{Beschreibung des Quellcodes}
Welches Modul macht was?

\chapter{Bedienungsanleitung}
Wie benutzt man das Teil? Screenshots mit Erklärung?

\chapter{Schlusswort}
Was haben wir gelernt? Hats uns Spaß gemacht?

%Flie"stext zur Arbeit. Schreiben Sie hier weiteren Text. Arbeiten
%Sie "uberall im Text mit den gewohnten \TeX{}-Befehlen.
%\nocite{*}% Dieser Befehl bewirkt, dass alle in der Bibliographie-Datei
% aufgef"uhrten Werke auch ausgegeben werden. Sonst werden nur diejenigen
% ausgegeben, die auch zitiert werden.

%\section{"Uberschrift der 2. Stufe}

%Weiterer Text.

%\subsection{"Uberschrift der 3. Stufe}

%Noch mehr Text

%\subsubsection{Letzte im Inhaltsverzeichnis ber"ucksichtigte "Uberschriftsebene}

%Und noch mehr Text.

\clearpage
\appendix%                   Einleitung der Anh"ange
% Die Angabe von \listoftables, index etc. funktioniert,
% wurde aber nicht ausdr"ucklich fuer die TUM angepa"st.
\listoffigures%              Abbildungsverzeichnis
%\bibliographystyle{tumbib} % TUM-spezifische Darstellung der Bibliographie
%\bibliography{arbeit}%       Name der .bib Datei mit der bibliographischen DB
%\clearpage
%\Ehrenwort%                  Ehrenw"ortliche Erkl"arung
\end{document}
