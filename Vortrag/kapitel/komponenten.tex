\section{Beschreibung der Komponenten}
\begin{frame}[<+->]{Beschreibung der Komponenten}
    \begin{figure}
        \centerline{
            \includegraphics[width=0.8\textwidth]{abbildungen/komponenten}
        }
    \end{figure}
\end{frame}
\begin{frame}[<+->]{Beschreibung der Komponenten}
    \begin{block}{Logikanalyzer}
        \begin{itemize}
            \item Steuerung der anderen Komponenten
            \item Speicherung der Signale mit Status des Logikanalyzers
            \item Auswerten der Taster und setzen der entsprechenden Signale
        \end{itemize}
    \end{block}
    \begin{block}{RAM}
        \begin{itemize}
            \item Dualport Block RAM
            \item Mit Core Generator erstellt
        \end{itemize}
    \end{block}
\end{frame}
\begin{frame}[<+->]{Beschreibung der Komponenten}
    \begin{block}{Sampler}
        \begin{itemize}
            \item Aufzeichnen der Messwerte im RAM
            \item Weitergabe der aktuellen Daten an Trigger
        \end{itemize}
    \end{block}
    \begin{block}{Sampler}
        \begin{itemize}
            \item Auswerten der aktuellen Messdaten
            \item Ausgabe eines Start Signals
        \end{itemize}
    \end{block}
\end{frame}
\begin{frame}[<+->]{Beschreibung der Komponenten}
    \begin{block}{VGA}
        \begin{itemize}
            \item Erzeugen des VGA Signals
            \item Eingabe: Alle Signale mit Statusinformationen
            \item Ergänzung von Flanken beim Zeichnen von Messwerten
        \end{itemize}
    \end{block}
    \begin{block}{VGA}
        \begin{itemize}
            \item Speicher für Schriftzeichen
            \item Schrift aus Linux Kernel mit 8x8 Pixel pro Zeichen
        \end{itemize}
    \end{block}
\end{frame}
