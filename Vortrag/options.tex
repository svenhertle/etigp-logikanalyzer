\documentclass[12pt]{beamer}
\usepackage[ngerman]{babel}
\usepackage[utf8]{inputenc}

% Thema
\usetheme{boxes}
% NOT \usecolortheme{fly}
\usefonttheme{structurebold}
%\useoutertheme{infolines}
\useoutertheme{infolines}

\beamertemplatenavigationsymbolsempty
%\setbeamertemplate{footline}[default]

% Weitere Pakete
\usepackage{amsmath,amsfonts,amssymb}
\usepackage{graphicx}
\usepackage{tabularx}
\usepackage{listings}
\usepackage{xcolor}

% Itemabstand vergrößern
\newlength{\wideitemsep}
\setlength{\wideitemsep}{\itemsep}
\addtolength{\wideitemsep}{1.5em}
\let\smallitem\item
\renewcommand{\item}{\setlength{\itemsep}{\wideitemsep}\smallitem}

\usepackage{textpos}
\setlength{\TPHorizModule}{1mm}
\setlength{\TPVertModule}{\TPHorizModule}

% C++
\def\CC{C\nolinebreak[4]\hspace{-.05em}\raisebox{.4ex}{\tiny\bf ++}\ }

% TODO Notiz
\newcommand\todo[1]{\textcolor{red}{TODO: #1}}

% Neue Befehle
\newcommand{\zB}{z.\,B.\ }
\newcommand{\dah}{d.\,h.\ }
\newcommand{\vgl}{vgl.\ }
\newcommand{\oae}{o.\,ä.\ }
\newcommand{\vlt}{vlt.\ }

% Listing
\lstset{ %
    language=C,                 % the language of the code
    basicstyle=\ttfamily,            % the size of the fonts that are used for the code
    breakatwhitespace=false,         % sets if automatic breaks should only happen at whitespace
    breaklines=true,                 % sets automatic line breaking
    stepnumber=2,                    % the step between two line-numbers. If it's 1, each line will be numbered
    tabsize=2,                       % sets default tabsize to 2 spaces
    morecomment=[l][\emph]{\#},
    captionpos=b,                    % sets the caption-position to bottom
}
